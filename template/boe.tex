\section{Bruisable Onion Encryption} \label{sec:definitions}
We introduce a new cryptographic primitive called bruisable onion encryption. Unlike in standard onion encryption, in bruisable onion encryption, each mixer on the routing path has a choice to add an extra bit of information to the onion: to ding (bruise) the onion or not. If the onion sustains too many bruises (i.e., a sufficient number of the mixers on the path bruise the onion), then the identity of the recipient $R$ and the innermost onion $O_\ell$ for the recipient become unrecoverable. 

Another difference between standard onion encryption and bruisable onion encryption is the addition of a new type of intermediaries, called \emph{gatekeepers}. 
A bruisable onion $O$ travels along its routing path $(M_1, \dots, M_{\ell_1}, G_1, \dots, G_{\ell_2}, R)$ consisting of some $\ell_1$ mixers, followed by some $\ell_2$ gatekeepers and the recipient~$R$. While the role of the mixers is to batch-process the onion (along with other onions) or to bruise it, gatekeepers are responsible for routing the onion all the way to the recipient only if the mixers didn't bruise it too much. 
\textcolor{red}{
Without gatekeepers, an onion that arrives at the last mixer $M_{\ell_1}$ on its routing path with the threshold number $X$ of bruises (with $X$ bruises the onion can be delivered to the recipient; with $X+1$, it cannot) can be processed by $M_{\ell_1}$ with or without further bruising. If $M_{\ell_1}$ is adversarial, it can try both and learn the number of bruises. So, we need gatekeepers to prevent this line of attack; an honest gatekeeper will detect that this is the same onion except for the number of bruises, and will not process it a second time. %While adversarial gatekeepers can delay the onions, this is not problematic because by that point, the onion (as long as it’s not bruised beyond the threshold) has already been sufficiently mixed with other onions. Lemma 2 in our security analysis addresses this situation.
}

%\subsection{I/O syntax}
A bruisable onion encryption scheme %is a $4$-tuple $(\keygen, \formonion, \proconion, \bruiseonion)$ 
consists of the following algorithms:  

\begin{description}
\item $\keygen$ takes the security parameter $\secparam$ and the name of a party $P$ as input, and outputs a public key pair $(\pk(P), \sk(P))$, i.e., 
$
(\pk(P), \sk(P)) \gets \keygen(\secparam, P). 
$

\item  $\formonion$ takes  
a (fixed length) message $m$, 
a routing path $\vec{P} = (M_1, \dots, M_{\ell_1},$ $G_1, \dots, G_{\ell_2}, R)$ 
consisting of $\ell_1$ ``mixers'' and $\ell_2$ ``gatekeepers,'' 
the public keys of the parties in $\vec{P}$, and 
a sequence $\vec{y} = (y_1, \dots, y_{\ell_1+\ell_2})$ of metadata where the metadata string $y_i$ is intended for the $i^\mathit{th}$ processing party on the routing path.  (The metadata conveyed to each intermediary is a useful component of an onion routing protocol: it allows the sender to communicate something about the onion to the processing party. For example, in our protocol in Section~\ref{sec:dp}, the metadata is the pseudorandom nonces in the checkpoint onions.)  
$\formonion$ outputs a list of lists of onions $\vec{O} = (\vec{O}_1, \dots \vec{O}_{\ell})$ where $\ell = \ell_1+\ell_2+1$. 
That is, letting $\pk(\vec{P})$ denote the public keys of the parties in $\vec{P}$, 
$
\vec{O} = (\vec{O}_1, \dots \vec{O}_{\ell}) \gets \formonion(m, \vec{P}, \pk(\vec{P}), \vec{y}). 
$

%\hspace{4mm} The metadata conveyed to each intermediary is a useful component of an onion routing protocol: it allows the sender to communicate something about the onion to the processing party.  Here, it can be useful to ensure that, once an honest party sends an onion $O_{i,j}$ to its next destination, it never processes another possible incarnation, $O_{i,j'}$ of the same onion: we will see that our protocol embeds a nonce in each layer of the onion.\todo{Metadata used for the checkpoint onions only in our construction.}

\hspace{4mm} In standard onion encryption as defined by Camenisch and Lysyanskaya~\cite{C:CamLys05}, $\formonion$ outputs a list of \emph{onions}, $O = (O_1, \dots, O_{\ell})$. This list is called the ``evolution of the onion'' because it is how the onion should evolve as it travels along the routing path; each $O_i$ is the onion that the $i^\mathit{th}$ intermediary should receive and process. 

\hspace{4mm} In bruisable onion encryption, the evolution depends on if and when the onion gets bruised. Accordingly, $\formonion$ outputs a list of lists of onions, $(\vec{O}_1, \dots \vec{O}_{\ell})$, where each list $\vec{O}_i$ contains all possible variations of the $i^\mathit{th}$ onion layer. 
The first list $\vec{O}_1 = (O_1)$ contains just the onion for the first mixer. 
For $2 \le i \le \ell_1$, the list $\vec{O}_i$ contains $i$ options, $\vec{O}_i = (O_{i, 0}, \dots, O_{i, i-1})$; each $O_{i, j}$ is what the $i^\mathit{th}$ onion layer should look like with $j$ prior bruises. 
For $\ell_1 + 1 \le i \le \ell_1 + \ell_2$, the list $\vec{O}_i$ contains $\ell_1+1$ options, depending on the total bruising from the mixers. 
The last list $\vec{O}_{\ell} = (O_{\ell_1+\ell_2+1})$ contains just the innermost onion for the recipient. 

\hspace{4mm} Note that the routing path $\vec{P}$ may start and/or end with a sub-path consisting of dummy parties, in which case $\formonion$ outputs onions for only the non-dummy routing parties. For example, if the routing path is $(\bot, \bot, P_3, P_4, P_5, \bot, \dots, \bot)$, $\formonion$ outputs $(\vec{O}_3, \vec{O}_4, \vec{O}_5)$. 

\item $\proconion$ takes the secret key $\sk(P)$ of the processing party $P$ and an onion~$O$. Its output is $(i, y, O',P')$ where $i$ is the position of the party $P$ on the onion's routing path and $y$ is the metadata, while $(O',P')$ falls into one of four cases: if $P$ is not the recipient, $(O',P')$ is either 
(1) the peeled onion $O'$ and its next destination $P'$ or (2) $(\bot,\bot)$ if the onion is malformed or too bruised; if $P$ is the recipient, then $P' = \bot$, while $O$ is either (3) a
message $m$ or (4) $\bot$.  
$
(i,y, O', P') \gets \proconion(\sk(P), O).
$

\item $\bruiseonion$ is an algorithm that allows an intermediary to damage the onion, or \textit{bruise} it.  This option is only available to the mixers on the routing path, i.e., to the first $\ell_1$ intermediaries.  $\bruiseonion$ takes as input the secret key $\sk(P)$ of the party $P$ and the onion~$O$ to be bruised as input, and outputs a bruised onion $O'$ to send to its next destination, 
$
O' \gets \bruiseonion(\sk(P), O).
$
\end{description}


\subsection{Correctness definition}
If a bruisable onion is processed only either by running the $\proconion$ algorithm or the $\bruiseonion$ algorithm at every hop, we require that it should travel along the intended routing path specified by the sender. Further, if the bruising isn't too bad (i.e., it falls under some threshold $\theta$), the gatekeepers should be able to recover the innermost onion and the recipient; otherwise, routing the onion through $(G_1, \dots, G_{\ell_2})$ should reveal the empty final destination $\bot$. We formalize this intuition below. 

%\begin{definition}
Let $\Sigma = (\keygen, \formonion, \proconion, \bruiseonion)$ be a bruisable encryption scheme. 
%
Let $\parties$ be any set of participants. 
%
For each $P_i \in \parties$, let $(\pk(P_i), \sk(P_i))$ be the key pair generated by running $\keygen$ on $P_i$. %, i.e., $(\pk(P_i), \sk(P_i)) \gets \keygen(\secparam, P_i)$. 
%
Let $m$ be any message from the message space; 
let $\vec{P} = (M_1, \dots, M_{\ell_1}$, $G_{1}, \dots, G_{\ell_2}, R)$ be any list of parties in $\parties$; 
let $\vec{y} = (y_1, \dots, y_{\ell_1+\ell_2})$ be any sequence of metadata. 
Let $\ell=\ell_1+\ell_2+1$. 
$(\vec{O}_1, \dots, \vec{O}_{\ell})$ is the result of running $\formonion$ on $m$, $\vec{P}$, the public keys $\pk(\vec{P})$ of the parties in $\vec{P}$, and $\vec{y}$, i.e., 
$
\vec{O} = (\vec{O}_1, \dots \vec{O}_{\ell}) \gets \formonion(m, \vec{P}, \pk(\vec{P}), \vec{y}).
$

We say that $\Sigma$ is correct with respect to the threshold $0 < \theta \le 1$, the number $\ell_1$ of mixers, and the number $\ell_2$ of gatekeepers if the following are satisfied: 

\begin{itemize}
\item \textbf{Correct peeling and bruising.} 
\sloppy 
For $1\leq i < \ell_1+\ell_2$, $1\leq j \leq |\vec{O}_i|$, let $(i',y,O,P)$ be the output of $\proconion(\sk(P_i),O_{i,j})$.  Then $i'=i$, $y=y_i$, $O=O_{i+1,j}$, and $P=P_{i+1}$.  In other words, when processing an onion, the mixer correctly recovers its position $i$ in the list of processing parties, its metadata $y_i$, the onion $O_{i+1,j}$ to send forth  with the same amount of bruising, and its destination $P_{i+1}$.  Moreover, for $1\leq i \le \ell_1$, $1\leq j \leq |\vec{O}_i|$, let $O'$ be the output of $\bruiseonion(\sk(P_i),O_{i,j})$.  Then $O' = O_{i+1,j+1}$.
 
\item \textbf{Correct gatekeeping.} For $i=\ell_1+\ell_2$, $1\leq j \leq \ell_1+1$, let $(i', y,O,P)$ be the output of $\proconion(\sk(P_i),O_{i,j})$.  If $j \leq \theta \ell_1$, then $i'=i$, $y=y_i$, $O=O_\ell$, and $P=R$.  In other words, when processing an onion that is not too bruised, the last gatekeeper correctly recovers its position $i=\ell_1+\ell_2$ in the list of processing parties, its metadata $y_i$, the onion $O_\ell$ to send forth, and the recipient $R$.  However, if $j > \theta \ell_1$,  then $i'=i$, $y=y_i$, $O=\bot$, and $P=\bot$.  In other words, if the onion is too bruised, the honest gatekeeper still recovers its metadata but not the onion to send forth or the next destination. 

\item \textbf{Correct message.} Peeling the innermost onion layer recovers the intended message, i.e., 
$
\proconion(\sk_{R}, O_{\ell}) = (\ell,\bot,m,\bot).
$
\end{itemize}
%\end{definition}












\iffalse 
\subsection{Bruisable onion security}
We define what it means for a bruisable onion encryption scheme to be secure. 

Let $\mathsf{BOSecurityGame}_{\secpar, \mathcal{OE}, \tau, \upsilon}^{\mathcal{A},b}$ be the following game parameterized by 
the security parameter $\secparam$, 
the bruisable onion encryption scheme $\mathcal{OE} = (\keygen, \formonion, \proconion, \bruiseonion)$, 
system parameters $\tau$ (controls how much bruising can be tolerated) and $\upsilon$ (controls the fraction of adversarial helper parties), 
the adversary $\adv$, and 
the challenge bit $b$. 

In this game, we will view the routing participants (senders, recipients and intermediaries on the onion's path) as distinct from helper participants (the ones that help the penultimate party on the onion's path to route it to its final destination).  

In the setup phase, the adversary picks the names of all the participants, including two honest routing parties $I_{\mathit{in}}$ and $I_{\mathit{out}}$.  The challenger runs $\keygen$ on behalf of $I_{\mathit{in}}$ and $I_{\mathit{out}}$; the adversary specifies the public keys for the rest of the routing participant.  The adversary controls an $\upsilon$ fraction of the helper parties.

In the first query phase, $\adv$ submits onions for processing to honest routing parties.  I.e., $\adv$ may submit a query to the challenger of the form $(O,P,a)$ where $P$ is an honest routing party with key pair $(\pk,\sk)$.  In response, $\adv$ receives $\proconion(\sk,O)$ if $a=0$, and $\bruiseonion(\sk,O)$ if $a=1$.  Note that in some circumstances (such as when $O$ is an honestly formed onion and $P$ is its penultimate routing party), $\proconion$ is a protocol initiated by $P$ and additionally involving honest helper parties.  $\adv$ may also query honest helper parties to initiate the $\proconion$ protocol, and may participate in it on behalf of adversarial helpers.

In the challenge phase, the adversary specifies the input to $\formonion$, that is, the values $m, \vec{I}, \vec{H}, \pk(\vec{I}\cup\vec{H})$ such that $\vec{I} = I_1,\ldots,I_\ell$ where $I_{\mathit{in}}=I_i$, $I_{\mathit{out}}=I_j$, $i < j$.  

If $b=0$, the onion is formed using $\formonion$ on input specified by $\adv$, and $O_1$ is returned to $\adv$.

If $b=1$, $((O_1), (O_{2, 0}, O_{2, 1}), \dots, (O_{i, 0}, \dots, O_{i, i-1}))= \formonion(\bot,(I_1,\ldots,I_i,\bot,\ldots,\bot),\bot,\pk(I_1,\ldots,I_i))$.  Return $O_1$.  Note: the way that $O_1$ is formed is independent of the onion's message, its routing path past $I_i = I_{\mathit{in}}$, and its helper set.  

In the second query phase, the adversary has the same capabilities to query honest routing parties as in the first query phase, unless $O$ was produced in the challenge phase, slated for processing by $I_{\mathit{in}}$ or $I_{\mathit{out}}$, or produced at an earlier point in the second query phase, slated for processing by $I_{\mathit{out}}$.  Here, we also disallow that $\adv$ query $I_{\mathit{in}}$ or $I_{\mathit{out}}$ with the challenge onion more than once; even though the challenge onion can look differently depending on how bruised it is, the routing participants will spot that it's the same onion and will only respond once.  The challenger also keeps a counter $c$ of how many times the onion has been bruised; the counter is initialized to $0$.

More precisely, if $\adv$'s query is a layer of the challenge onion, i.e., $(O_{r,k},I,a)$ such that $(r,I) \in \{(i,I_\mathit{in}),(j,I_\mathit{out})\}$ and $\adv$ has already issued a query $(O_{r,k'},I,a')$ in the past for $k'\neq k$ or $a\neq a'$, then output $\bot$ (this is the case that the same onion has already been queried).  Next, if $b=0$, process the onion according to $\proconion$ if $a=0$, and according to $\bruiseonion$ otherwise.

Else (i.e., if $b=1$): increment the counter $c$ to note how many times the onion was bruised so far, i.e., $c := c+k+a$.  Next, the challenger creates the next layer(s) of the onion.  If $I = I_\mathit{in}$ and $i < \ell-1$, obtain $((O_{i+1, 0}, \dots, O_{i+1, i}), \dots, (O_{j, 0}, \dots, O_{j,j-1})) = \formonion(\bot,(\bot,\ldots,\bot,I_{i+1},\ldots,I_{j},\bot,\ldots,\bot),\bot,\pk(I_{i+1},\ldots,I_j))$, and return to the adversary the onion $O_{i+1,0}$.  

If $i = \ell-1$, and $I_{\mathit{out}}$ is the recipient, so depending on whether the helpers are honest and how much bruising there has been, either get an onion to $I_{\mathit{out}}$ or not.  FOUR CASES  [Must add instructions for how to bruise a challenge onion]

If $I=I_{\mathit{out}}$ and $j < \ell-1$, obtain $\vec{O} = \formonion(\bot,(\bot,\ldots,\bot,I_{j+1},\ldots,I_{\ell-1},\bot),\vec{H},\pk(I_{j+1},\ldots,I_{\ell-1}, H_1,\ldots,H_h))$, and return to the adversary the onion $O_{j+1,0}$.  (If helpers are honest, then recipient and message are hidden at this point; but if not, then they are not hidden.)

If $j = \ell-1$, run the protocol with the helpers, and if they are honest and not too much bruising, get the onion for the recipient.  

[Must add instructions for how $I_j$ bruises a challenge onion]

If $j = \ell$, return $m$.

If $\adv$ queries the helpers $\vec{H}$ with onion $O_{\ell-1,k}$ and the helpers are honest majority, then check if $k_{\mathit{out}}+k$ is too much bruising.  If it is too much, the helpers output $\bot$; otherwise they run $\formonion(m,\bot,...,\bot,I_\ell)$ and output $(O_{\ell,1},I_\ell)$.

\todo{Fill in.}
\fi